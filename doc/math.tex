\documentclass[11pt,twocolumn]{amsart} % twocolumn
\usepackage{geometry}                % See geometry.pdf to learn the layout options. There are lots.
\geometry{a4paper}                   % ... or a4paper or a5paper or ... 
\usepackage{layout}
%\geometry{landscape}                % Activate rotated page geometry
%\usepackage[parfill]{parskip}    % Activate to begin paragraphs with an empty line rather than an indent
\usepackage{graphicx}
\usepackage{amssymb}
\usepackage{epstopdf}
\DeclareGraphicsRule{.tif}{png}{.png}{`convert #1 `dirname #1`/`basename #1 .tif`.png}
\DeclareMathOperator*{\argmin}{arg\,min}

% shortcuts
\newcommand{\ve}[1]{\boldsymbol{#1}}
\newcommand{\ma}[1]{\boldsymbol{#1}}
\newenvironment{m}{\begin{bmatrix}}{\end{bmatrix}}


\title{Some stuff for \textbf{home}}
\author{Gaspard Buma}
\date{}                                           % Activate to display a given date or no date

\begin{document}

\twocolumn[
\maketitle
\emph{Voici quelques notes sur les math\'{e}matiques utilis\'{e}es pour le spectacle \textbf{home}. Le texte est en anglais parce que les accents sont difficiles \`{a} \'{e}crire avec Latex et puis la plupart des articles et livres que je lis sur le sujet sont en anglais.}
]
\section{Notations}
In this text, the following typographic notations has been used:
\begin{align*}
  \text{importance of frequency f} = X(f) = & \int_{-\infty}^{+\infty}x(t)e^{-i2\pi ft}\,dt \\
																					= & \int_{-\infty}^{+\infty}x(t) (\cos(2\pi ft)+i\sin(2\pi ft)) \,dt \\
					\text{removing imaginary part}  = & \int_{-\infty}^{+\infty}x(t) \cos(2\pi ft) \,dt 
\end{align*}
\end{document}